\documentclass{book}

\usepackage[english]{babel}
\usepackage{amssymb}
\usepackage{amsmath}
\usepackage{txfonts}
\usepackage{mathdots}
\usepackage[classicReIm]{kpfonts}
\usepackage[pdftex]{graphicx}



\begin{document}

\noindent \begin{flushleft}
Chapter two

\noindent 

\noindent constructed by human beings. As such, ?nding useful and accurate information on the

\noindent Net requires a certain amount of skill as well as access to a variety of research and

\noindent retrieval engines.

\noindent 

\noindent REFERENCES

\noindent 

\noindent December ,J.(1994), Challenges for a webbed society. Computer-Mediated Communication Magazine,

\noindent l$\mathrm{\{}$8). [online]. Available: http://www.december.com/cmc/mag/1994/ov/websoc.html.

\noindent Diamond, E., \& Bates, S. (1995). The ancient history of the Internet. American Heritage. 46, 34-45.

\noindent Jackson , M. (1997) . Assessing the  structure of communication on the World Wide Web. Journal of

\noindent Computer Mediated Communication, (1). [Online] . Available: http://www.ascusc.org/jcmc/vol3/

\noindent Issue 1/Jackson.html.

\noindent Marchionini , G . (1988). Hypermedia and learning: Freedom and chaos. Educational Tecbnology,28(11),  8-12.

\noindent Nelson, T. H. (1967). Getting it out of our system. In G. Schechter (Ed.),Information retrieval: A critical review (pp. 191-210). Washington, DC:  Thompson.

\noindent Sterling, B. (1993). A short history of the Internet by  Bruce  Sterling. [Online]. Available: http://w3

\noindent .aces.uiuc.edu/ALM/scale/nethistory.r.html.

\noindent Underwired. (1997). History of the Internet. [Online]. Available: http://www.underwired.com/report/uw.css.

\noindent 
\end{flushleft}


\end{document}

