\documentclass{book}

\usepackage[english]{babel}
\usepackage{amssymb}
\usepackage{amsmath}
\usepackage{txfonts}
\usepackage{mathdots}
\usepackage[classicReIm]{kpfonts}
\usepackage[pdftex]{graphicx}



\begin{document}



\noindent \begin{flushleft}
WHAT IS THE NET?

\noindent 

\noindent Listen to sound segments and use proprietary meta tag systems to selectively

\noindent search through their large collections of songs and music.

\noindent 

\noindent An excellent and always available starting point for spelling, translation, thesaurus,

\noindent and encyclopedia references, and occasional relevant links is the program

\noindent Atomic a (http://www.atomica.ca) This free program, when installed on your

\noindent machine, provides instant (ALT CLICK) access to any word in your browser,

\noindent word processing document, or any other text program---very handy!

\noindent 

\noindent Other services function by quickly submitting your single request to multiple

\noindent search engines and displaying the results on a single (long!) screen. For example

\noindent Dog Pile (www.Dogpile.com) "fetches" the results of your search from sixteen different

\noindent search engines.

\noindent 

\noindent Finally, most search engines allow the user to turn on a ``kiddie filter'' to eliminate 

\noindent  adult content hits, and some such as Surf Safely (http://www.surfsafely.com/)

\noindent or Family Safe Startup Page (http://www.startup-page.com/) index only Family-

\noindent rated sites.

\noindent 

\noindent Once you have decided which search tool to use, refer to that system's online help section.

\noindent  Not all search engines work the same, nor do they have the same features and

\noindent commands. Moreover, there is nearly always advice and tutorials available that can be

\noindent extremely useful when refining your search. In the end, taking the nine to read the

\noindent online help documentation will save you time by achieving better search results. Also,

\noindent keep in mind that rarely will you be satis?ed with your First search results. Be prepared

\noindent to try a variety of combinations of Boolean searches and synonyms for keywords.

\noindent Finally, all search engines and subject guides have different methods of re?ning queries

\noindent and/or retrieving information. The best way to learn them is to read the help ?les on

\noindent the search engine sites, and don't be afraid to experiment!

\noindent 

\noindent TIPS FOR FINDING USEFUL AND RELEVANT INFORMATION

\noindent 

\noindent 

\noindent Begin your search by analyzing your needs.

\noindent Isolate your keywords.

\noindent Select a search tool that matches your needs.

\noindent Experiment with a variety of search engines and subject guides! 

\noindent Although our favorite search engines tend to change over time, we have found

\noindent Google (http://www.google.com) to be the fastest, most effective, and least ad-

\noindent cluttered site.

\noindent 

\noindent 

\noindent SUMMARY

\noindent 

\noindent The Net has evolved into an unstructured network of millions of computers through-

\noindent out the world. Today most of us access the Net through the use of a suite of protocols

\noindent known as the World Wide Web (WWW). Besides being described as the most complicated

\noindent  network, the Internet has also been described as the largest network ever

\noindent 
\end{flushleft}


\end{document}

