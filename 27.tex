\documentclass{book}

\usepackage[english]{babel}
\usepackage{amssymb}
\usepackage{amsmath}
\usepackage{txfonts}
\usepackage{mathdots}
\usepackage[classicReIm]{kpfonts}
\usepackage[pdftex]{graphicx}


\begin{document}


\noindent \begin{flushleft}
CHAPTER THREE                                                                                                                                       

\noindent 

\noindent DESIGNING e-RESEARCH

\noindent 

\noindent It is the theory that decides what can be observed.

\noindent Albert Einstein 

\noindent 

\noindent Because the Net is a large, multipurpose, evolving tool, determining its best use in

\noindent any research application is a challenging task. However, the Net is also Famous for

\noindent spurring innovation at "Internet speed." Frequently leaving authors of paper books

\noindent struggling to keep up. In this chapter we discuss what is perhaps the most important

\noindent and challenging task of the e-researcher---to design research that asks meaningful and

\noindent answerable questions and that coherently answers these questions in ways that match

\noindent the personal worldview of the researcher, the sponsor of the research, and the subjects

\noindent of investigation.

\noindent 

\noindent Considerable research is being conducted using the Internet as a data-gathering,

\noindent analysis, and dissemination tool, even though the advantages and disadvantages of

\noindent using the Internet for these purposes remain relatively unexplored. Often, those using

\noindent the Net do so with little guidance with respect to what kind of research data is most

\noindent appropriately collected online. Based on work  by early adopters of e-research, it would

\noindent appear that when the researcher has a good understanding of the Net (including its

\noindent culture and technological limitations and advantages) that almost any kind of research

\noindent could be effectively adapted. Further, when creatively approached and thoughtfully

\noindent designed, research can be conducted and disseminated using the Net with a number of

\noindent notable advantages, which are discussed in the last section of this chapter. This being

\noindent said, there are circumstances under which the Net will be of little or no use to the

\noindent research process. At one time, for example, the Net was only useful for observing

\noindent activities that took place on it. Now, however, Net-based surveys, focus groups, inter-

\noindent views, and unobtrusive Web cameras (Webcams) allow researchers to observe and 

\noindent collect data about events that take place both on and off the Net.

\noindent 

\noindent Much of the research in the social sciences and education focuses on processes

\noindent that cannot be seen and measured with external and quanti?able tools (e.g., the internal

\noindent mental processes of learning). Since these processes are invisible, it takes the innovative 

\noindent  skills of the researcher to develop both Net and non-Net techniques to understand

\noindent 
\end{flushleft}


\end{document}

